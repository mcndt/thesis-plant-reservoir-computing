
% Need for deterministic computations
Strictly speaking, for a reservoir to possess reproducible computational properties, it must show a deterministic relation between its input sequence and its state \citep{nakajima_physical_2020}. 
Put plainly, applying the same input sequence should reliably produce the same sequence of output states.


% Determinism is too strict for physical systems
This precondition is particularly strict on physical systems because it requires precise control over the initial conditions of the physical reservoir (Equation \ref{prc:update}).
This requirement is virtually unattainable as it is impossible to account for all physical factors of an environment that influence the reservoir before the start of the experiment.
Fortunately, a lesser but sufficient condition exists to perform useful computation with a reservoir.


% The echo state property comes to the rescue
The echo state property (\acrshort{esp}) (also referred to as the fading memory property) defines a less strict precondition under which it is possible to derive computations from a dynamical system.
Informally, the \acrshort{esp} states that a sufficient condition for reservoir computing is for the input echo function $\phi$ (Equation \ref{prc:echo_input_function}) to have fading memory of past inputs \citep{lukosevicius_reservoir_2012}. 
If this assumption is valid, then the trajectories of the reservoir state will converge when subjected to the same input sequence, regardless of the initial conditions.
A side-effect of the echo state property is that it limits the reservoir's memory of past events, thus limiting the available integration window for memory-bound tasks.


% Example in ESN.
Figure \ref{fig:fading_mem_principle} demonstrates the principle using an ESN. It shows the evolution of a single node in the network for two runs. 
At step 0, we initialized the reservoir state with a different random state vector. 
After step 0, we subject the reservoir to the same input sequence for both scenarios.
While the initial trajectory depends on the node's starting value, the neuron synchronizes to the input sequence entirely by step 75. 
The node's evolution no longer depends on the initial value from that point onwards.



% Fading memory can be recognized in many systems
Fading memory can be recognized in physical dynamical systems as well.
Oftentimes it is the result of a dampening force such as electrical resistance or drag.
For example, \citet{nakajima_information_2015} implemented \acrshort{prc} using a silicone octopus arm submerged in water. 
In their particular case, the. fluid resistance of the water tank exerts a dampening effect on the silicone arm, resulting in a fading memory.


% The mathematical reason why this works is still being researched
The understanding behind this property is that the response of many dynamical systems eventually synchronizes to the input signal, suppressing chaotic behavior \citep{nakajima_physical_2020}. 
However, the mathematical understanding of the \acrshort{esp} and its relation to dynamical systems is still a topic of active research (see e.g.\ \cite{yildiz_re-visiting_2012, lu_attractor_2018}).
