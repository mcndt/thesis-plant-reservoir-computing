
% Introduction
To recap Chapter \ref{chapter:lit-prc}: to solve a computational problem with a physical medium, it must possess two fundamental properties.
The first is linear input separation relative to the target signal.
A reservoir can achieve this by performing a nonlinear expansion of its environmental inputs.
To perform tasks that require memory, the reservoir must also possess recurrent dynamics; 
the reservoir integrates current inputs with its memory of the past.
The second requirement is a fading memory of past inputs.
Without a fading influence of past inputs, computations carried out by the reservoir cannot consistently be reproduced.
The question remains: do plants display these properties?


% Input separation and nonlinearity
First, we consider linear input separation.
To make a target a linear function of the reservoir state, the reservoir must perform some nonlinear transformation of its inputs.
An example of this in plants is how grass responds to soil temperature. 
The growth rate of grass shows a roughly parabolical relationship with soil temperature, reaching an optimum around roughly 15-20\degree C, after which growth declines due to heat stress \citep{hurtado-uria_relationships_2013}.
This shows that a plant's behavior can show a nonlinear relation with its environment.


% Fading memory
Next, we consider the fading memory property.
Clearly, this rule does not hold for the general case; a plant temporarily subjected to extreme cold or heat will die and no longer process any information.
Favorable environmental conditions will influence the plant's growth rate, permanently altering reservoir dynamics as a function of past inputs.
Nonetheless, we might be able to identify operating ranges for specific eco-physiological processes and time scales for which the fading memory property holds up approximately.


% Reconciling dynamic reservoirs with the static reservoir assumption
In addition to the properties above, the \acrshort{prc} framework assumes the reservoir dynamics are unchanging over time.
Naturally, changing the reservoir's constitution also changes its nonlinear characteristics and memory capacity.
Plants are not stationary reservoirs in the general case because they go through several growth phases that change their structure and behavior drastically.
But here, too, we can approximate stationarity by investigating reservoir dynamics within a time window for which the growth rate is negligible.


% Conclusion
In conclusion, there is sufficient evidence to suggest that plants can display computational properties for eco-physiologically relevant tasks.
However, plant RC research is still in its infancy. It remains unexplored under which conditions the RC framework applies to plant behavior and what subset of eco-physiological processes show reservoir-like dynamics.


\subsection{Research Objectives}

% Objectives
\citet{pieters_reservoir_2022} has already demonstrated that plants can function as a medium for reservoir computing.
This work aims to expand on the foundation for plant \acrshort{rc}. 
We aim to answer the following questions:
From the subset of plant physiological processes that are empirically observable, which processes show promising input separation for eco-physiological tasks and computational benchmarks? 
And do these properties display the fading memory principle at the relevant time scale?
To answer these questions, we propose a framework for quantitatively measuring the reservoir properties of the plant's physiological functions.
This framework enables direct comparison of the reservoir performance of different physiological processes and across multiple plants.


% Why use in silico plants?
We used \acrshort{fspm}s to test this framework. 
Using simulated plant has several advantages.
It gives us the freedom to observe the plant directly, without the need for sensing technology and laboratory setups.
Consequently, it speeds up the research process and gives us a first intuition into suitable plant reservoirs. 
This can help bootstrap future \textit{in vivo} research.
In addition, we believe that \acrshort{rc} can contribute to the plant modeling community by identifying potential weaknesses in \acrshort{fspm}s.
The reason is that RC can be used for analyzing short-term as well as large scale variations; whereas current modeling efforts focuses only on long term behavior,
even though research shows that rapid physiological responses play an essential role in plant life \citep{alarcon_substantial_1994, deswaef2015a}.
