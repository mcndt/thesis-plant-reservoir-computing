
\chapter{Conclusion} \label{chapter:discussion}

% **Introduction:**
The previous chapter reported the quantitative results of our experiments along with our interpretation of the data and a discussion of their implications.
This chapter offers broader conclusions to the experiments we conducted.
% We also point out potential weaknesses in our experimental setup and how they could be mitigated in future work.
Then, we briefly recap the weaknesses we found in mechanistic plant models and how they can be improved.
Finally, we propose a few directions for future research and potential applications of plant RC.


% **Results implications:**
\section{Implications of the Results}

Our experiments showed promising results for plant RC with several plant physiological processes.
% We observed similar results for each of the processes we considered.
To pick reservoirs for future research, we recommend cross-referencing our results of reservoir performance with available sensing technologies to select the physiological reservoir with the highest potential.
For example, surface temperature, transpiration rate, and stomatal conductance form a good combination of reservoir features because they can all be observed using thermal cameras (see \mbox{Table \ref{table:imaging-techniques}}).


% **Better FSPMs:**
\section{Improving Mechanistic Plant Modeling}

Our research with FSPMs also highlighted some blind spots in mechanistic plant modeling.
Most importantly, we see a need for developing mechanistic models that operate at a much smaller time resolution, i.e.\ minute- or second-scale.
Another area that deserves further attention is modeling chaotic behavior in plant physiological processes.
Understanding chaos in reservoirs and studying how input signals can amplify or suppress chaotic behavior is paramount to the successful application of physical reservoir computing \citep{nakajima_physical_2020}.
The formulas and methods in this book can be adapted into a toolset for validating the behavior of plant models.
For example, \mbox{Equation \ref{methods:reservoir_divergence}} can be used to measure the chaos present in the plant model quantitatively.  


% **Future work:**
\section{Future Work}

% *Expanding on the foundation:*
The study of using plants as a substrate for reservoir computing is still in its infancy.
There are several yet unexplored topics to extend the foundation of plant RC.
For example, we currently only have data on RC with three species: strawberry plants from \citet{pieters_reservoir_2022}, and now \textit{in silico} versions of grapevine and wheat.
Further research must establish a broader basis for predicting the reservoir properties of a species, e.g. by studying the role of the size of the plant.
Another area that requires attention is how RC can be reconciled with plant growth.
For example, we can use online learning approaches such as Hebbian learning or methods inspired by neuromorphic systems to fit readout functions that update as the plant's physiological behavior changes.

% *Moving towards applications:*
Looking towards applications, there are numerous exciting paths to explore.
For example, plant RC could detect abiotic stresses such as heat or drought stress.
Reservoir computing could also be applied in phenotyping to characterize the plant's physiological response to various environments.
Even more ambitious, plant RC could be used to design a closed-loop system that optimizes greenhouse conditions for vegetative growth and crop yield.
We could use reward-modulated Hebbian learning techniques \citep{burms_reward-modulated_2015} to create a self-learning control system that adapts to the needs of the plant as it grows.
Here we can recommend the continued use of CN-Wheat \citep{barillot_cn-wheat_2016} and WheatFspm \citep{gauthier_functional_2020} for an \textit{in silico} proof-of-concept. 
These FSPMs model vegetative and grain growth, respectively, and showed stable behavior in our experiments.


% **Conclusion:**
% \section{Conclusion}
\vspace{0.5cm}

We believe that the results from this work form good additional evidence that plants can be used as a substrate for physical reservoir computing.
Our results can serve as a guide for future research in plant RC, both \textit{in vivo} and \textit{in silico}.
This is just the starting point for plant reservoir computing. 
If knowledge and technology can scale, there will be countless opportunities in agricultural technology.
