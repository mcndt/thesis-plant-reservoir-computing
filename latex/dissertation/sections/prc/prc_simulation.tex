If the behavior of the $\phi$-function is understood completely, it is possible to create a computer model that simulates the reservoir dynamics \citep{nakajima_physical_2020}. 
Using a simulated reservoir can have various advantages in research. 
(i) It is possible to iterate faster on designing synthetic physical reservoirs, such as FPGAs or photonic circuits, by validating designs in software before constructing the physical reservoir.
(ii) Some reservoirs operate at long time scales \citep{ushio_computational_2021}. Simulation can accelerate the research cycle by running experiments faster than in real-time.
(iii) In simulation, it is possible to operate in a highly controlled environment, which makes it possible to solve a constrained version of a problem first. 
For example, it is possible to dictate identical initial conditions in every experiment run or enforce static reservoir dynamics where the physical counterpart may experience some malleability caused by the environment.