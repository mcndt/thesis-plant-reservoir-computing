
% Introduction
In this section, we propose our framework for quantitatively measuring input separation and fading memory properties of eco-physiological processes in plants.
With these methods we will compare the capabilities of different processes.
The results will highlight which observable plant properties to pursue in future \acrshort{rc} research.


\subsection{Input Separation}

% **FIGURES:**
% - Table of regression tasks to carry out (or not)

% Introduction
First, we measure the linear input separation of the reservoir w.r.t.\ various target tasks.
We propose a collection of regression targets based on the environment and physiological state of the plant.
We choose these tasks to test the reservoir's capacity to solve biological problems.
Classification tasks are not considered because they are less relevant from an eco-physiological standpoint; 
most processes are real-valued and occur in continuous time.
We fit a simple linear readout model to the reservoir observations to test the prediction accuracy.
We then compare the predicted series against the ground truth data to compute an accuracy metric.


% Regression tasks
To paint a complete picture of the reservoir, we use a mix of target tasks.
We propose three categories of targets, previously established by \citet{pieters_reservoir_2022}: (i) environmental inputs, (ii) eco-physiological state, and (iii) computational benchmarks.
Predicting environmental inputs gives us a first indication of the plant's coupling with the environment.
For example, we can use ambient temperature, incident solar radiation or relative humidity as a prediction target.
Next, eco-physiological tasks indicate whether the observed reservoir is effective for solving biological problems.
Targets in this category can be, e.g.\ transpiration rate or photosynthesis rate on the organism level.
Finally, computational benchmarks measure a reservoir's degree of nonlinearity and memory.
We propose three targets: a delay line to measure memory capacity (\mbox{Equation \ref{prc:delay-line-benchmark}}), a polynomial expansion to measure nonlinearity (Equation \ref{prc:polynomial-benchmark}), and the NARMA benchmark to measure both properties (\mbox{Equation \ref{prc:narma-timescale-adapted}}).
Each artificial target uses an environmental series as input.
These targets are not practical tasks for a plant, but the results can help contextualize plants in the broader \acrshort{prc} literature.


% Readout model
We fit a linear readout model for every reservoir-target combination.
For the readout model we propose a simple linear regression model with a bias term:

\begin{equation}
\hat{y}[t] = \mathbf{W}^{\text{out}}\left[1;\mathbf{X}[t] \right] \label{methods:readout}
\end{equation}

To isolate the computational capacity of the reservoir from the natural correlation between the target and the environment, the environmental inputs are not fed into the readout model; the prediction relies on the reservoir only.
Model parameters \(\mathbf{W}^{\text{out}}\) are fitted to training data using ridge regression (\mbox{Equation \ref{esn:training}}).
The regularization \(\lambda\) parameter is tuned for each reservoir-target pairing. 
To find the optimal \(\lambda\), we propose a parameter sweep with logarithmic spacing, validated using cross-validation on the training data.

% Evaluation metric
We evaluate the reservoir performance on a set of held-out test data. 
Prediction accuracy is measured using the \acrfull{nmse} metric:

\begin{equation}
\text{NMSE} = \frac{1}{N} \sum_{t=1}^{N} \frac{\left(y[t] - \hat{y}[t]\right)^{2}}{\text{var}(y)} \label{methods:regression_nmse}
\end{equation}

A lower score means a more accurate prediction.
The \acrshort{nmse} has several advantages over regular \acrshort{mse} \citep{pieters_reservoir_2022}.
Because it is normalized, the results can be compared across reservoir-target pairings, including between plants.
It is also easy to interpret: a perfect predictor scores 0.0, while predicting the signal mean for every step yields a score of 1.0.


\subsection{Fading Memory}

% Introduction 
To test fading memory, we propose the impulse experiment described in Section \ref{sec:fading_memory} and illustrated in Figures \ref{fig:memory_input_impulse} and \ref{fig:memory_esn_impulse}.
The impulse can be applied to any of the environmental inputs.
The amplitude and duration of the impulse should remain within realistic boundaries so that an actual plant would not suffer permanent damage that alters the reservoir dynamics.


% Evaluation metric
To quantitatively measure the divergence $\delta(t)$ between two reservoir trajectories at a given time step, we propose a modified \acrshort{nmse} metric:

\begin{equation} \label{methods:reservoir_divergence}
    \delta(t) = \frac{1}{N} \sum_{i=1}^{N} \frac{\left( \mathbf{X}_i^{C}(t) - \mathbf{X}_i^{E}(t) \right)^{2}}{\text{var}\left( \mathbf{X}_i^{C} \right)}
\end{equation}

Where $N$ is the size of the observed reservoir, $\mathbf{X}^{C}$ the reservoir state in the control experiment, and $\mathbf{X}^{E}$ the reservoir state during the impulse experiment.
We can use a line plot of the divergence during and after the impulse to inspect the fading memory property of the reservoir.
A reservoir affected by the impulse should show a peak in divergence once the impulse is applied.
If the physiological process displays fading memory, the divergence should go down to pre-impulse levels in a finite amount of time after the artificial stimulus is removed.
Note that a substantial divergence between reservoirs subjected to the same inputs also indicates the echo state property may not hold.