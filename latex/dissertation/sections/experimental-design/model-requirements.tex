
Many physiological processes inside the plant can be considered as a reservoir.
To ensure the results from this work can form the foundation for future research, we direct our attention towards those processes which are observable using existing sensing technologies.
In his doctoral dissertation, \citet{pieters_reservoir_2022} compiled comprehensive tables of physiological traits that are observable using imaging techniques (Table \ref{table:imaging-techniques}) and contact-based techniques (Table \ref{table:non-imaging-techniques}).
We can use these tables to choose which processes to look for when selecting plant models.

This work shows a preference for reservoirs observable by contactless imaging systems.
In the future, contactless systems will play a key role in scaling plant \acrshort{rc} to large arrays of plants because they do not influence reservoir dynamics and can observe multiple plants at once.
However, imaging technologies for plant RC also have some drawbacks.
They are more susceptible to noise induced by ambient factors than contact-based sensors.
Dynamic ambient factors must also be accounted for when calibrating the measurements.
Image-based systems tend to have a lower resolution, capturing less subtle variations than contact sensors \citep{pieters_reservoir_2022}.


% - Many eco-physiological processes can be considered as reservoirs. 
% - However, to keep the results of this work relevant for future research, we want to focus on processes that are observable using existing sensing technologies.
% 	- With a preference for contactless technologies; in the future we want to be able to scale PRC to larger arrays of plants on the roadmap towards practical applications in e.g. greenhouse settings.
% - In his doctoral dissertation, Pieters (2022) compiled comprehensive tables of physiological traits observable using contactless imaging techniques (table 3.1) and contact-based techniques (table 4.2).
% - For his own work he observed leaf thickness as the reservoir using leaf clips.
% 	- Leaf thickness strongly correlates with water status and turgor pressure (de swaef 2015).
% 	- However, for our own work using FSPMs, we should be able to access physiological traits directly.

