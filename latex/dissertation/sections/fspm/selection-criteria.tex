
% Types of plant models
% Plant models exist in many shapes and sizes. 
There exist many different types of plant models.
To get an idea of the various model categories one can visit \textit{Quantitative Plant}\footnote[1]{https://www.quantitative-plant.org/model. See \citet{lobet_online_2013, lobet_image_2017} for more details.}, 
a website that aggregates open-source plant models from the literature.
Listed models range from crop surface models (\acrshort{csm}) that estimate biomass and crop yield from aerial imagery \citep{tilly_multitemporal_2014} to sub-plant models that simulate gas exchange between the canopy and the environment \citep{sinoquet_ratp_2001}.

%%% TODO %%%

% Selection criteria: single organism
To select plant models for use in \acrshort{rc} experiments, they must meet certain criteria. 
Firstly, the model must track the state of a single plant over time, 
as the scope of the research presented limits the interpretation of the reservoir to a single organism.
This rules out the use of e.g.\ crop surface models because they only inform about large populations of plants; in these models information on the level of an individual organism is lost (although it would certainly be interesting to apply reservoir computing principles to the ecological dynamics of a crop field).


% Selection criteria: 3d anatomy
Secondly, the plant's state should be observable from multiple fixed points of the plant's anatomy. 
These observed points must be distinct structural elements of the plant that simulate the physiological processes of a real specimen and must communicate with one another as a dynamical system.
We set this requirement because an RC system using a plant model as reservoir is intended to represent the equivalent setup in the real world.
This requirement entails a preference for mechanistic models that make few prior assumptions about plant behavior, leaving macroscopic behavior to emerge naturally from microscopic interactions at the cellular level \citep{gauthier_functional_2020}.
Ideally, the observable parameters of the plant are also empirically measurable \textit{in vivo} to enable the research results from this work to be better transferable to a real-world setting.


% Selection criteria: time resolution
Finally, the time resolution of the simulation should be sufficiently fine to observe dynamics at multiple time scales as well as avoid signal aliasing (see \mbox{Section \ref{section:rc-time-scales}}).

