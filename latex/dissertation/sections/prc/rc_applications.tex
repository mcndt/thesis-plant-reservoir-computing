
% Introduction to application domains.
Reservoir computing has enjoyed success in a multitude of research domains because of its simple computational model, the potential for real-time processing, and broad applicability to problems in the physical world \citep{tanaka_recent_2019}. 
Early applications of RC include time series prediction tasks such as stock price forecasting and weather forecasting \citep{nakajima_physical_2020}. 
Table \ref{table:rc-subjects} shows a comprehensive summary of the domains to which RC has been applied as compiled by \citet{tanaka_recent_2019}.
Similarly, \mbox{Table \ref{table:rc-applications-benchmarks}} gives a selection of works that demonstrate the application of reservoir computing. 


\begin{table}[!ht]
    \raggedleft
    \caption[Examples of subjects in RC applications.]{Examples of subjects in RC applications. This table originally appeared in \citet{tanaka_recent_2019} and has been reused under the CC BY 4.0 license.}
    % \def\arraystretch{1.4}%  1 is the default, change whatever you need
    \begin{tabular}{p{0.25\linewidth}  p{0.7\linewidth}}
    \toprule
        \textbf{Category} & \textbf{Examples} \\ \midrule
        Biomedical & EEG, fMRI, ECG, EMG, heart rates, biomarkers, BMI, eye movement, mammogram, lung images. \\ 
        \arrayrulecolor{black!10!white}
        \midrule
        Visual & Images, videos. \\ 
        \midrule
        Audio & Speech, sounds, music, bird calls. \\ 
        \midrule
        Machinery & Vehicles, robots, sensors, motors, compressors,  controllers, actuators. \\ 
        \midrule
        Engineering & Power plants, power lines, renewable energy, engines, fuel cells, batteries, gas flows, diesel oil, coal mines, hydraulic excavators, steam generators, roller mills, footbridges, air conditioners. \\ 
        \midrule
        Communication & Radio waves, telephone calls, Internet traffic. \\ 
        \midrule
        Environmental & Wind power and speed, ozone concentration, PM2.5, wastewater, rainfall, seismicity. \\ 
        \midrule
        Security & Cryptography. \\ 
        \midrule
        Financial & Stock price, stock index, exchange rate. \\ 
        \midrule
        Social & Language, grammar, syntax, smart phone. \\ 
    \arrayrulecolor{black}
    \bottomrule
    \end{tabular}
    \label{table:rc-subjects}
\end{table}



% TODO: make this table prettier (2022-02-17)
% TODO: add references in this table to bibliography (2022-02-17)
\begin{table}[!ht]
    \raggedleft
    \caption[Applications and related benchmark tasks of RC.]{
        Applications and related benchmark tasks of RC. This table originally appeared in \citet{tanaka_recent_2019} and has been reused under the CC BY 4.0 license.
    }
    % \def\arraystretch{1.4}%  1 is the default, change whatever you need
    \begin{tabular}{p{0.25\linewidth}  p{0.7\linewidth}}
    \toprule
        \textbf{Applications} & \textbf{Benchmark tasks} \\ 
        \midrule
        Pattern classification & Spoken digit recognition (Verstraeten, Schrauwen, Stroobandt and Van Campenhout, 2005), 
        Waveform classification (Paquot et al., 2012), 
        Human action recognition (Soh \& Demiris, 2012),
        Handwritten digit image recognition (Jalalvand, Van Wallendael, \& Van de Walle, 2015) \\
        \arrayrulecolor{black!10!white}
        \midrule
        Time series forecasting & Chaotic time series prediction (Jaeger, 2001),
        NARMA time series prediction (Jaeger, 2003) \\
        \midrule
        Pattern generation & Sine-wave generation (Jaeger, 2002), 
        Limit cycle generation (Hauser, Ijspeert, Füchslin, \& Maass, 2012) \\
        \midrule
        Adaptive filtering and control & Channel equalization (Jaeger \& Haas, 2004) \\
        \midrule
        System approximation & Temporal XOR task (Bertschinger \&; Natschläger, 2004), 
        Temporal parity task (Bertschinger \&; Natschläger, 2004) \\
        \midrule
        Short-term memory & Memory capacity (Jaeger, 2002) \\
    \arrayrulecolor{black}
    \bottomrule
    \end{tabular}
    \label{table:rc-applications-benchmarks}
\end{table}

% Tables \ref{table:rc-subjects} and \ref{table:rc-applications-benchmarks} give an overview of subjects to which RC has been applied and references to applications and benchmarks respectively.
% In robotics, RC is deployed as a low-power real-time integrator of sensor data and motor controller  \citep{nakajima_information_2015, burms_reward-modulated_2015}.



% \ref{table:rc-subjects}

% One of the earliest demonstrations of reservoir computing is the Liquid State Machine (2002)\citep{maass_real-time_2002}.

% Since its inception, reservoir computing has been successfully applied in several domains. % though today it has mostly been succeeded by Deep Learning techniques \cite{citation needed}.

% - Example of LSM \cite{maass_real-time_2002}

% - Examples of applications are \cite{lukosevicius_reservoir_2012, nakajima_physical_2020}. see also Tanaka 2019 section 2.2 table 1 and table 2 \cite{tanaka_recent_2019}.

% - Also the source of more theoretical research in mathematics and chaos, see \cite{hermans_recurrent_2012, gauthier_next_2021}

 