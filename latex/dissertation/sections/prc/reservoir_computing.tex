% Where does RC come from?
Reservoir computing (\acrshort{rc}) is a machine learning method proposed in the early 2000s as an alternative to training recurrent neural networks (\acrshort{rnn}) \citep{jaeger_echo_2002}. 
Training weights in recurrent networks requires computationally expensive and highly linear training algorithms such as backpropagation-through-time (\acrshort{bptt}). 
Moreover, \acrshort{bptt} often suffers from numerical instability and often yields suboptimal solutions \citep{bengio_learning_1994}.

In the \acrshort{rc} framework, the untrained \acrshort{rnn} (called the reservoir) is interpreted as a black box dynamical system.
Input signals are transformed by the nonlinear activations of the neurons and are integrated with the past state of the reservoir through recurrent connections \citep{jaeger_harnessing_2004}.
As a result, the reservoir displays both nonlinear behavior and memory capacity.
We can then exploit these properties to solve a variety of regression tasks.


% What does an RC model look like?
\mbox{Figure \ref{fig:rc_diagram}} shows a diagram of a general \acrshort{rc} model using an \acrshort{rnn} as the reservoir.
To accomplish regression tasks, an \acrshort{rc} model consists of two parts: the nonlinear reservoir that transforms environmental inputs and a linear readout function that observes the reservoir to perform a given task. 
The linear readout function observes the state of the reservoir and applies a linear regression model to determine the final output. 
When there is no feedback from the readout function back to the reservoir, it is possible to train multiple readouts that perform different tasks based on the same reservoir dynamics.

\begin{figure}[t]
	\centering
    \includesvg[width=0.65\textwidth]{img/rnn-illustration.svg}
    \vspace{0.35cm}
    % \includegraphics[width=0.5\textwidth]{example-image}
	\caption[A reservoir computing model with a RNN as the reservoir, commonly referred to as an echo state network (ESN).]
	        {A reservoir computing model with a RNN as the reservoir, commonly referred to as an echo state network (ESN).
    	 	 Figure reproduced from \citet{pieters_reservoir_2022} with permission from the author.}
	\label{fig:rc_diagram}
\end{figure}


% What makes RC work?
The key to understanding reservoir computing is that all non-linearity and memory required to solve a given task are already present in the reservoir dynamics. 
The readout function then learns to distill the information present in the observed reservoir state to solve a given task. 
In this way, \acrshort{rc} is similar to the kernel method from classical machine learning, where the reservoir acts as a nonlinear expansion and temporal convolution of the input signals \citep{hermans_recurrent_2012}. 
However, kernel techniques do not account for time explicitly, whereas reservoir computing inherently occurs in the time domain. 
This strong connection to the time domain makes \acrshort{rc} particularly useful for solving tasks with time series data and real-time applications.


% RC is data-efficient and easier to train.
In its original form, the neurons inside the reservoir are randomly connected and are generally not trained. 
However, there is literature that proposes task-independent tuning schemes that improve the stability of the reservoir \citep{lukosevicius_reservoir_2009, norton_preparing_2006, tanaka_effect_2020}.
Reservoir computing models are notably more data-efficient and easier to train because there is no need to train the recurrent part of the network using complex and data-hungry algorithms such as \acrshort{bptt}.


% Concrete frameworks for implementing reservoir computing.
The literature has proposed several implementations of reservoirs.
The echo state network (\acrshort{esn}), as proposed by \citet{jaeger_echo_2002}, is a purely mathematical implementation where neuron activations are determined as a nonlinear function of the weighted sum of incoming connections. 
\acrshort{esn}s operate in discrete time only and are ideal for software implementations.
Subsection \ref{subsection:esn} gives a detailed mathematical description of the echo state network.
Independently, \citet{maass_real-time_2002} proposed the liquid state machine (\acrshort{lsm}), an idea that originated from computational neuroscience.
In this model, the recurrent network is formed by spiking neurons that can operate in both discrete and continuous time.
The neurons receive spike trains (discrete events in time) as input and generate an output sequence of spikes accordingly.
A third approach proposed by \citet{steil_backpropagation-decorrelation_2004} is called Backpropagation-Decorrelation (\acrshort{bpdc}). 
It is a learning algorithm for online training of the readout weights.


\subsection{Echo State Network} \label{subsection:esn}


% introduction
The Echo State Network (ESN) is a mathematical framework that implements the core concepts of reservoir computing \citep{jaeger_tutorial_2002}. 
In this framework, the reservoir is a discrete-time RNN using a sigmoidal nonlinearity (such as the tanh function). 
The activation weights of the reservoir are initialized randomly and remain fixed. 
The linear readout function takes the reservoir state as input and is fit to the desired target signal.

The reservoir state in time step $t$ is calculated from the input signal $\mathbf{u}$ using the following update rules \citep{lukosevicius_reservoir_2012},


\begin{align}
    \tilde{\mathbf{x}}[t] &= \tanh \left( \mathbf{W}^{\text{in}}[1;\mathbf{u}[t]] + \mathbf{W} \mathbf{x}[t-1] \right) \label{esn:update1} \\
    \mathbf{x}[t] &= (1-\alpha) \mathbf{x}[t-1] + \alpha \tilde{\mathbf{x}}[t] \label{esn:update2}
\end{align}


where $\mathbf{x}$ signifies the reservoir neurons' activations, $\mathbf{\tilde{x}}$ the activation update, and $\alpha$ a scalar leaking rate. The readout function $\mathbf{y}$ is fulfilled by a simple linear regression of the reservoir state and an optional direct feed-in of the input signal.

\begin{equation}
\mathbf{y}[t] = \mathbf{W}^{\text{out}}\left[1; \mathbf{u}[t];\mathbf{x}[t] \right] \label{esn:readout}
\end{equation}


The weights $\mathbf{W}^{\text{out}}$ of the readout function can be trained using a variety of learning methods. A common offline approach for fitting the function output to a target signal is ridge regression \citep{lukosevicius_reservoir_2012}:

\begin{equation}
    \mathbf{W}^{\text{out}} = \mathbf{Y}_{\text{target}}\mathbf{X}^{T} \left(\mathbf{X} \mathbf{X}^{T} + \lambda^{2} \mathbf{I} \right)^{-1} \label{esn:training}
\end{equation}

where $\lambda$ is a regularization parameter that prevents overfitting.
For $\lambda$ = 0, the model attempts to find weights that fit the training data exactly.
The results is a high training accuracy, but low accuracy on unseen data.
Large model weights are penalized for $\lambda$ > 0, resulting in smaller weights and a more generalized model.
However, if $\lambda$ is too large, the model parameters shrink too much, and the model loses expressiveness.
The optimal value for $\lambda$ should be selected using validation data separate from training and test data.
Other regression techniques are also possible.
For example, \citet{burms_reward-modulated_2015} have demonstrated online training of the readout function using reward-modulated Hebbian plasticity rules.


\subsection{Applications} \label{subsection:rc_applications}


% Introduction to application domains.
Reservoir computing has enjoyed success in a multitude of research domains because of its simple computational model, the potential for real-time processing, and broad applicability to problems in the physical world \citep{tanaka_recent_2019}. 
Early applications of RC include time series prediction tasks such as stock price forecasting and weather forecasting \citep{nakajima_physical_2020}. 
Table \ref{table:rc-subjects} shows a comprehensive summary of the domains to which RC has been applied as compiled by \citet{tanaka_recent_2019}.
Similarly, \mbox{Table \ref{table:rc-applications-benchmarks}} gives a selection of works that demonstrate the application of reservoir computing. 


\begin{table}[!ht]
    \raggedleft
    \caption[Examples of subjects in RC applications.]{Examples of subjects in RC applications. This table originally appeared in \citet{tanaka_recent_2019} and has been reused under the CC BY 4.0 license.}
    % \def\arraystretch{1.4}%  1 is the default, change whatever you need
    \begin{tabular}{p{0.25\linewidth}  p{0.7\linewidth}}
    \toprule
        \textbf{Category} & \textbf{Examples} \\ \midrule
        Biomedical & EEG, fMRI, ECG, EMG, heart rates, biomarkers, BMI, eye movement, mammogram, lung images. \\ 
        \arrayrulecolor{black!10!white}
        \midrule
        Visual & Images, videos. \\ 
        \midrule
        Audio & Speech, sounds, music, bird calls. \\ 
        \midrule
        Machinery & Vehicles, robots, sensors, motors, compressors,  controllers, actuators. \\ 
        \midrule
        Engineering & Power plants, power lines, renewable energy, engines, fuel cells, batteries, gas flows, diesel oil, coal mines, hydraulic excavators, steam generators, roller mills, footbridges, air conditioners. \\ 
        \midrule
        Communication & Radio waves, telephone calls, Internet traffic. \\ 
        \midrule
        Environmental & Wind power and speed, ozone concentration, PM2.5, wastewater, rainfall, seismicity. \\ 
        \midrule
        Security & Cryptography. \\ 
        \midrule
        Financial & Stock price, stock index, exchange rate. \\ 
        \midrule
        Social & Language, grammar, syntax, smart phone. \\ 
    \arrayrulecolor{black}
    \bottomrule
    \end{tabular}
    \label{table:rc-subjects}
\end{table}



% TODO: make this table prettier (2022-02-17)
% TODO: add references in this table to bibliography (2022-02-17)
\begin{table}[!ht]
    \raggedleft
    \caption[Applications and related benchmark tasks of RC.]{
        Applications and related benchmark tasks of RC. This table originally appeared in \citet{tanaka_recent_2019} and has been reused under the CC BY 4.0 license.
    }
    % \def\arraystretch{1.4}%  1 is the default, change whatever you need
    \begin{tabular}{p{0.25\linewidth}  p{0.7\linewidth}}
    \toprule
        \textbf{Applications} & \textbf{Benchmark tasks} \\ 
        \midrule
        Pattern classification & Spoken digit recognition (Verstraeten, Schrauwen, Stroobandt and Van Campenhout, 2005), 
        Waveform classification (Paquot et al., 2012), 
        Human action recognition (Soh \& Demiris, 2012),
        Handwritten digit image recognition (Jalalvand, Van Wallendael, \& Van de Walle, 2015) \\
        \arrayrulecolor{black!10!white}
        \midrule
        Time series forecasting & Chaotic time series prediction (Jaeger, 2001),
        NARMA time series prediction (Jaeger, 2003) \\
        \midrule
        Pattern generation & Sine-wave generation (Jaeger, 2002), 
        Limit cycle generation (Hauser, Ijspeert, Füchslin, \& Maass, 2012) \\
        \midrule
        Adaptive filtering and control & Channel equalization (Jaeger \& Haas, 2004) \\
        \midrule
        System approximation & Temporal XOR task (Bertschinger \&; Natschläger, 2004), 
        Temporal parity task (Bertschinger \&; Natschläger, 2004) \\
        \midrule
        Short-term memory & Memory capacity (Jaeger, 2002) \\
    \arrayrulecolor{black}
    \bottomrule
    \end{tabular}
    \label{table:rc-applications-benchmarks}
\end{table}

% Tables \ref{table:rc-subjects} and \ref{table:rc-applications-benchmarks} give an overview of subjects to which RC has been applied and references to applications and benchmarks respectively.
% In robotics, RC is deployed as a low-power real-time integrator of sensor data and motor controller  \citep{nakajima_information_2015, burms_reward-modulated_2015}.



% \ref{table:rc-subjects}

% One of the earliest demonstrations of reservoir computing is the Liquid State Machine (2002)\citep{maass_real-time_2002}.

% Since its inception, reservoir computing has been successfully applied in several domains. % though today it has mostly been succeeded by Deep Learning techniques \cite{citation needed}.

% - Example of LSM \cite{maass_real-time_2002}

% - Examples of applications are \cite{lukosevicius_reservoir_2012, nakajima_physical_2020}. see also Tanaka 2019 section 2.2 table 1 and table 2 \cite{tanaka_recent_2019}.

% - Also the source of more theoretical research in mathematics and chaos, see \cite{hermans_recurrent_2012, gauthier_next_2021}

 