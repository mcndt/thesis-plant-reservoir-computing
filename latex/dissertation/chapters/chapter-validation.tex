This chapter will contain experimental proof that plants behave as reservoirs, using FSPMs, based on i.a. Pieters et al. 2021 \supercite{pieters_plants_2021} and Ushio et al. 2021 \supercite{ushio_computational_2021}.

\section{Goals}

\begin{itemize}
\item What are the key metrics we need to evaluate of our reservoir?
\item How can the results of this preliminary evaluation impact further research?
\item I.e. better understanding of the strengths and shortcomings of various FSPMs, which should be well explored before proving any reservoir computing hypothesis that rely on the FPSM as an accurat representation of the plant it simualtes.
\item What is our expectation? (i.e. wish to observe at least the same performance as Olivier’s, if the FSPM is realistic)
\end{itemize}


\section{Experimental setup}

\begin{itemize}
\item Which FSPMs have been explored
\item How do we observe the in silico plant
\item How do we collect time series
\item Training the readout function
\item Train/test/evaluation split + grouping strategy
\item Evaluation tests and metrics
\end{itemize}

\section{Results}


\begin{itemize}
\item Charts that indicate the performance of the FSPMs evaluated
\item Effect of reservoir size, time delay, heterogeneous state observations, ...
\end{itemize}

\section{Discussion}

\begin{itemize}
\item Were the results as expected?
\item What did we learn about these FSPMs?
\item What are the key takeaways to take into account in further research?
\end{itemize}

\section{Conclusion}

\begin{itemize}
\item Key considerations to carry forward in the interpretation of the results in this body of work.
\end{itemize}

