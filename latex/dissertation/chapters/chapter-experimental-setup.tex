\chapter{Experimental Design} \label{chapter:exp-setup}

% Introduction
 Chapter \ref{chapter:lit-prc} established the necessary background on \acrshort{rc} and explained how physical media can function as a computational resource.
Then, Chapter \ref{chapter:fspm} introduced the rich research field of functional-structural plant modeling.
Now, we present how we will identify reservoir dynamics in plants and how we can quantitatively measure the reservoir performance, using simulated plants as the experiment subject.


\section{Reservoir Dynamics in Plants}

% Introduction
To recap Chapter \ref{chapter:lit-prc}: to solve a computational problem with a physical medium, it must possess two fundamental properties.
The first is linear input separation relative to the target signal.
A reservoir can achieve this by performing a nonlinear expansion of its environmental inputs.
To perform tasks that require memory, the reservoir must also possess recurrent dynamics; 
the reservoir integrates current inputs with its memory of the past.
The second requirement is a fading memory of past inputs.
Without a fading influence of past inputs, computations carried out by the reservoir cannot consistently be reproduced.
The question remains: do plants display these properties?


% Input separation and nonlinearity
First, we consider linear input separation.
To make a target a linear function of the reservoir state, the reservoir must perform some nonlinear transformation of its inputs.
An example of this in plants is how grass responds to soil temperature. 
The growth rate of grass shows a roughly parabolical relationship with soil temperature, reaching an optimum around roughly 15-20\degree C, after which growth declines due to heat stress \citep{hurtado-uria_relationships_2013}.
This shows that a plant's behavior can show a nonlinear relation with its environment.


% Fading memory
Next, we consider the fading memory property.
Clearly, this rule does not hold for the general case; a plant temporarily subjected to extreme cold or heat will die and no longer process any information.
Favorable environmental conditions will influence the plant's growth rate, permanently altering reservoir dynamics as a function of past inputs.
Nonetheless, we might be able to identify operating ranges for specific eco-physiological processes and time scales for which the fading memory property holds up approximately.


% Reconciling dynamic reservoirs with the static reservoir assumption
In addition to the properties above, the \acrshort{prc} framework assumes the reservoir dynamics are unchanging over time.
Naturally, changing the reservoir's constitution also changes its nonlinear characteristics and memory capacity.
Plants are not stationary reservoirs in the general case because they go through several growth phases that change their structure and behavior drastically.
But here, too, we can approximate stationarity by investigating reservoir dynamics within a time window for which the growth rate is negligible.


% Conclusion
In conclusion, there is sufficient evidence to suggest that plants can display computational properties for eco-physiologically relevant tasks.
However, plant RC research is still in its infancy. It remains unexplored under which conditions the RC framework applies to plant behavior and what subset of eco-physiological processes show reservoir-like dynamics.


\subsection{Research Objectives}

% Objectives
\citet{pieters_reservoir_2022} has already demonstrated that plants can function as a medium for reservoir computing.
This work aims to expand on the foundation for plant \acrshort{rc}. 
We aim to answer the following questions:
From the subset of plant physiological processes that are empirically observable, which processes show promising input separation for eco-physiological tasks and computational benchmarks? 
And do these properties display the fading memory principle at the relevant time scale?
To answer these questions, we propose a framework for quantitatively measuring the reservoir properties of the plant's physiological functions.
This framework enables direct comparison of the reservoir performance of different physiological processes and across multiple plants.


% Why use in silico plants?
We used \acrshort{fspm}s to test this framework. 
Using simulated plant has several advantages.
It gives us the freedom to observe the plant directly, without the need for sensing technology and laboratory setups.
Consequently, it speeds up the research process and gives us a first intuition into suitable plant reservoirs. 
This can help bootstrap future \textit{in vivo} research.
In addition, we believe that \acrshort{rc} can contribute to the plant modeling community by identifying potential weaknesses in \acrshort{fspm}s.
The reason is that RC can be used for analyzing short-term as well as large scale variations; whereas current modeling efforts focuses only on long term behavior,
even though research shows that rapid physiological responses play an essential role in plant life \citep{alarcon_substantial_1994, deswaef2015a}.



\section{Framework For Measuring Plant Reservoir Dynamics}

% Introduction
In this section, we propose our framework for quantitatively measuring input separation and fading memory properties of eco-physiological processes in plants.
With these methods we will compare the capabilities of different processes.
The results will highlight which observable plant properties to pursue in future \acrshort{rc} research.


\subsection{Input Separation}

% **FIGURES:**
% - Table of regression tasks to carry out (or not)

% Introduction
First, we measure the linear input separation of the reservoir w.r.t.\ various target tasks.
We propose a collection of regression targets based on the environment and physiological state of the plant.
We choose these tasks to test the reservoir's capacity to solve biological problems.
Classification tasks are not considered because they are less relevant from an eco-physiological standpoint; 
most processes are real-valued and occur in continuous time.
We fit a simple linear readout model to the reservoir observations to test the prediction accuracy.
We then compare the predicted series against the ground truth data to compute an accuracy metric.


% Regression tasks
To paint a complete picture of the reservoir, we use a mix of target tasks.
We propose three categories of targets, previously established by \citet{pieters_reservoir_2022}: (i) environmental inputs, (ii) eco-physiological state, and (iii) computational benchmarks.
Predicting environmental inputs gives us a first indication of the plant's coupling with the environment.
For example, we can use ambient temperature, incident solar radiation or relative humidity as a prediction target.
Next, eco-physiological tasks indicate whether the observed reservoir is effective for solving biological problems.
Targets in this category can be, e.g.\ transpiration rate or photosynthesis rate on the organism level.
Finally, computational benchmarks measure a reservoir's degree of nonlinearity and memory.
We propose three targets: a delay line to measure memory capacity (\mbox{Equation \ref{prc:delay-line-benchmark}}), a polynomial expansion to measure nonlinearity (Equation \ref{prc:polynomial-benchmark}), and the NARMA benchmark to measure both properties (\mbox{Equation \ref{prc:narma-timescale-adapted}}).
Each artificial target uses an environmental series as input.
These targets are not practical tasks for a plant, but the results can help contextualize plants in the broader \acrshort{prc} literature.


% Readout model
We fit a linear readout model for every reservoir-target combination.
For the readout model we propose a simple linear regression model with a bias term:

\begin{equation}
\hat{y}[t] = \mathbf{W}^{\text{out}}\left[1;\mathbf{X}[t] \right] \label{methods:readout}
\end{equation}

To isolate the computational capacity of the reservoir from the natural correlation between the target and the environment, the environmental inputs are not fed into the readout model; the prediction relies on the reservoir only.
Model parameters \(\mathbf{W}^{\text{out}}\) are fitted to training data using ridge regression (\mbox{Equation \ref{esn:training}}).
The regularization \(\lambda\) parameter is tuned for each reservoir-target pairing. 
To find the optimal \(\lambda\), we propose a parameter sweep with logarithmic spacing, validated using cross-validation on the training data.

% Evaluation metric
We evaluate the reservoir performance on a set of held-out test data. 
Prediction accuracy is measured using the \acrfull{nmse} metric:

\begin{equation}
\text{NMSE} = \frac{1}{N} \sum_{t=1}^{N} \frac{\left(y[t] - \hat{y}[t]\right)^{2}}{\text{var}(y)} \label{methods:regression_nmse}
\end{equation}

A lower score means a more accurate prediction.
The \acrshort{nmse} has several advantages over regular \acrshort{mse} \citep{pieters_reservoir_2022}.
Because it is normalized, the results can be compared across reservoir-target pairings, including between plants.
It is also easy to interpret: a perfect predictor scores 0.0, while predicting the signal mean for every step yields a score of 1.0.


\subsection{Fading Memory}

% Introduction 
To test fading memory, we propose the impulse experiment described in Section \ref{sec:fading_memory} and illustrated in Figures \ref{fig:memory_input_impulse} and \ref{fig:memory_esn_impulse}.
The impulse can be applied to any of the environmental inputs.
The amplitude and duration of the impulse should remain within realistic boundaries so that an actual plant would not suffer permanent damage that alters the reservoir dynamics.


% Evaluation metric
To quantitatively measure the divergence $\delta(t)$ between two reservoir trajectories at a given time step, we propose a modified \acrshort{nmse} metric:

\begin{equation} \label{methods:reservoir_divergence}
    \delta(t) = \frac{1}{N} \sum_{i=1}^{N} \frac{\left( \mathbf{X}_i^{C}(t) - \mathbf{X}_i^{E}(t) \right)^{2}}{\text{var}\left( \mathbf{X}_i^{C} \right)}
\end{equation}

Where $N$ is the size of the observed reservoir, $\mathbf{X}^{C}$ the reservoir state in the control experiment, and $\mathbf{X}^{E}$ the reservoir state during the impulse experiment.
We can use a line plot of the divergence during and after the impulse to inspect the fading memory property of the reservoir.
A reservoir affected by the impulse should show a peak in divergence once the impulse is applied.
If the physiological process displays fading memory, the divergence should go down to pre-impulse levels in a finite amount of time after the artificial stimulus is removed.
Note that a substantial divergence between reservoirs subjected to the same inputs also indicates the echo state property may not hold.


\section{Considered Reservoirs} \label{sec:considered-reservoirs}

Many physiological processes inside the plant can be considered as a reservoir.
To ensure the results from this work can form the foundation for future research, we direct our attention towards those processes which are observable using existing sensing technologies.
In his doctoral dissertation, \citet{pieters_reservoir_2022} compiled comprehensive tables of physiological traits that are observable using imaging techniques (Table \ref{table:imaging-techniques}) and contact-based techniques (Table \ref{table:non-imaging-techniques}).
We can use these tables to choose which processes to look for when selecting plant models.

This work shows a preference for reservoirs observable by contactless imaging systems.
In the future, contactless systems will play a key role in scaling plant \acrshort{rc} to large arrays of plants because they do not influence reservoir dynamics and can observe multiple plants at once.
However, imaging technologies for plant RC also have some drawbacks.
They are more susceptible to noise induced by ambient factors than contact-based sensors.
Dynamic ambient factors must also be accounted for when calibrating the measurements.
Image-based systems tend to have a lower resolution, capturing less subtle variations than contact sensors \citep{pieters_reservoir_2022}.


% - Many eco-physiological processes can be considered as reservoirs. 
% - However, to keep the results of this work relevant for future research, we want to focus on processes that are observable using existing sensing technologies.
% 	- With a preference for contactless technologies; in the future we want to be able to scale PRC to larger arrays of plants on the roadmap towards practical applications in e.g. greenhouse settings.
% - In his doctoral dissertation, Pieters (2022) compiled comprehensive tables of physiological traits observable using contactless imaging techniques (table 3.1) and contact-based techniques (table 4.2).
% - For his own work he observed leaf thickness as the reservoir using leaf clips.
% 	- Leaf thickness strongly correlates with water status and turgor pressure (de swaef 2015).
% 	- However, for our own work using FSPMs, we should be able to access physiological traits directly.




\section{Summary}

This chapter proposed a framework for quantitatively measuring reservoir characteristics of plant-physiological behavior.
In the first test, we use a suite of regression targets and a standard readout model to measure the capacity to perform nonlinear tasks.
% The predicted series is compared to the ground truth using the \acrshort{nmse} metric.
The second test investigates the fading memory property in an experiment where a brief artificial stimulus is applied to the plant's environment.
% We measure how the plant dynamics restore to their original trajectory using a divergence measure based on \acrshort{nmse}.
For the considered reservoirs, we prefer observable physiological traits using contactless sensing technologies. 
Still, the selection of reservoirs shown in the results will depend on what is available in the selected plant models.

% - Briefly recap how we will qualitatively and quantitatively measure reservoir performance
% 	- input separation experiments
% 	- fading memory experiments
% - Briefly recap what specific requirements this needs of our plant models.



\begin{sidewaystable}[hbpt]
    \centering
    \caption[Imaging techniques for various plant-physiological processes.]{Imaging techniques for various plant-physiological processes. From \citet{pieters_reservoir_2022}: ``Depending on sensor type, different target traits can be investigated. Moreover, the timescale of the physiological process is also indicated. When measuring on this timescale, we can observe noticeable variation in that particular trait. It ranges days (d), hours (h) to minutes (min) and seconds (s).'' Table reused from \citet{pieters_reservoir_2022} with permission from the author.}
    \label{table:imaging-techniques}
    % \tiny

    % \def\arraystretch{1.2}
    \begin{tabular}{llll}
        \toprule
        \textbf{sensor}                                    & \textbf{target trait}                                   & \textbf{timescale}                 & \textbf{reference}                               \\
        \midrule
        \multirow{6}{6cm}{thermal camera}                  & leaf temperature $T_\text{leaf}$                        & s→min & \textcite{costa2013}                             \\
                                                           & stomatal conductance (\textleftarrow $T_\text{leaf}$)   & s→min & \textcite{jones1999}                             \\
                                                           & transpiration (\textleftarrow $T_\text{leaf}$)          & s→min & \textcite{maes2012}                              \\
                                                           & drought stress (\textleftarrow $T_\text{leaf}$)         & d                          & \textcite{deswaef2021}                           \\
                                                           & diseases and pathogens (\textleftarrow $T_\text{leaf}$) & h→d      & \textcite{costa2013}                             \\
                                                           & heat stress (\textleftarrow $T_\text{leaf}$)            & min→h   & \textcite{janka2013}                             \\
        \midrule
        % depth camera/\glsxtrshort{LiDAR}                                 & 3D plant architecture                                   & h→d      & \textcite{busemeyer2013}, \textcite{friedli2016} \\
        depth camera/\glsxtrshort{LiDAR}                                 & 3D plant architecture                                   & h→d      & \textcite{busemeyer2013}, \textcite{friedli2016} \\
        \midrule
        \multirow{6}{6cm}{\glsxtrshort{RGB} camera}                      & leaf area growth                                        & h→d      & \textcite{walter2007}                            \\
                                                           & plant height                                            & h→d      & \textcite{borra-serrano2019}                     \\
                                                           & plant biomass                                           & h→d      & \textcite{golzarian2011}                         \\
                                                           & plant size and architecture                             & h→d      & \textcite{lootens2016}                           \\
                                                           & drought stress (reflectance and biomass)                & d                          & \textcite{mazis2020}                             \\
                                                           & diseases and pathogens (reflectance)                    & h→d      & \textcite{chaerle2007}                           \\
        \midrule
        \multirow{8}{6cm}{chlorophyll fluorescence imager} & leaf photosynthesis                                     & min→h   & \textcite{baker2008}                             \\
                                                           & stomatal conductance                                    & s→min & \textcite{nejad2005}                             \\
                                                           & chilling stress                                         & min→h   & \textcite{devacht2011}                           \\
                                                           & drought stress                                          & d                          & \textcite{meyer1999}                             \\
                                                           & diseases and pathogens                                  & min→h   & \textcite{berger2007}                            \\
                                                           & heat stress                                             & h                         & \textcite{briantais1996}                         \\
                                                           & salt stress                                             & h→d      & \textcite{moradi2007}                            \\
                                                           & nutrient deficiency                                     & d                          & \textcite{ciompi1996}                            \\
        \midrule
        \multirow{7}{6cm}{hyperspectral camera}            & leaf pigments                                           & d                          & \textcite{blackburn2007}                         \\
                                                           & canopy density                                          & d                          & \textcite{carlson1997}                           \\
                                                           & photosynthesis (PRI)                                    & min→h   & \textcite{inoue2008}                             \\
                                                           & water content                                           & h→d      & \textcite{whetton2018}                           \\
                                                           & drought stress                                          & d                          & \textcite{berger2010}                            \\
                                                           & diseases and pathogens                                  & h→d      & \textcite{bock2010,vandevijver2020}              \\
                                                           & nutrient deficiency                                     & d                          & \textcite{zhao2005}                              \\
        \bottomrule
    \end{tabular}
    
\end{sidewaystable}


\begin{sidewaystable}[hbpt]
    \centering
    \caption[Non-imaging techniques for various plant-physiological processes.]{Non-imaging techniques for various plant-physiological processes. From \citet{pieters_reservoir_2022}: ``Depending on sensor type, different target traits can be investigated. Moreover, the timescale is also indicated in which measurements with noticeable variation can be recorded, ranging from days (d), hours (h) to minutes (min) and seconds (s).'' Table reused from \citet{pieters_reservoir_2022} with permission from the author.}
    \label{table:non-imaging-techniques}
    % \tiny
    % \def\arraystretch{1.2}
    \begin{tabularx}{0.8\textwidth}{lllX}
        \toprule
        \textbf{sensor} & \textbf{target trait} & \textbf{timescale} & \textbf{reference}  \\
        \midrule
        \multirow{3}{5.2cm}{weighing scale} & transpiration & s→min & \textcite{wallach2010}  \\
                                    & plant biomass & d & \textcite{vanleperen1994} \\
                                    & drought stress (biomass)  & d & \textcite{wallach2010}  \\
        \arrayrulecolor{black!10!white}
        \midrule
        \multirow{2}{5.2cm}{\glsxtrfull{LVDT}} & leaf length & s→min  & \textcite{barillot2020} \\
                                      & stem diameter & s→min & \textcite{deswaef2015a} \\
        \midrule
        leaf clip & leaf thickness       & s→min    & \textcite{deswaef2015}      \\
        \midrule
        stem psychrometer & stem water potential & s→min      & \textcite{vandegehuchte2014} \\
        \midrule
        Zim-probe & leaf turgor pressure & s→min      & \textcite{rodriguez-dominguez2019} \\
        \midrule
        acoustic & formation of cavitations & s→min      & \textcite{debaerdemaeker2019} \\
        \midrule
        electrical impedance spectroscopy & water uptake & s→min & \textcite{ehosioke2020} \\
        electrical impedance tomography & root biomass & s→min & \textcite{ehosioke2020} \\
        \midrule
        \multirow{2}{5.2cm}{sap flow} & water uptake & min & \multirow{2}{5.2cm}{\textcite{hanssens2015}} \\
                        & transport rate & min & \\
        \arrayrulecolor{black}
        \bottomrule
    \end{tabularx}
    
\end{sidewaystable}