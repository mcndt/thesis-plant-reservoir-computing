We present a framework to evaluate the computational capabilities of plant physiological processes in plant models for physical reservoir computing (PRC).
PRC is a computing paradigm that enables the use of physical substrates for computation. 
% Recently, the use of live plants in a PRC context was demonstrated. 
Recently, Pieters (2022) proposed and successfully demonstrated the use of plants in a PRC context.
We expand upon this work \textit{in silico} using plant models for grapevine and wheat. 
The use of computer simulations gave us the freedom to observe the plant directly, without the need for sensing technology and laboratory setups.
In our experiments, we investigated the use of leaf temperature, transpiration rate, photosynthesis rate, light absorption, water flow, and hydraulic pressure as readout traits. 
Each of these traits displayed reservoir dynamics and correlated well with biologically relevant tasks. 
Surprisingly, our framework highlighted unusual behavior in mechanistic plant models, which the authors did not yet report.
We propose that the plant-modeling community can adapt the methods in this work to verify the behavior of plant models. 
We believe that this is just the starting point for plant reservoir computing.
If knowledge and technology can scale, there will be countless opportunities in agricultural technology.




% % **Goals:**
% Physical reservoir computing (PRC) is a machine learning method that exploits physical materials to perform computations.
% PRC has enjoyed success in domains such as soft body robotics and photonic circuits.
% In this thesis, we continue his work.
% We set out to identify which plant physiological processes display emergent computational qualities.
% To this end, we present a framework for making a quantitative comparison of plant physiological reservoirs.
% % **Methods:**
% We applied this framework with \textit{in silico} plants;
% we used existing mechanistic models of grapevine and common wheat found in the literature.
% Consequently, it sped up the research process and gave us a first intuition about suitable plant reservoirs.
% Our experiments evaluated the computational properties of leaf temperature, transpiration rate, photosynthesis rate, light absorption, water flow, and hydraulic pressure.
% % **Results:**
% Each of these physiological processes showed reservoir dynamics and correlated well with biologically relevant tasks.
% As a side effect, our framework uncovered flaws and unusual behavior in the mechanistic plant models that were previously unreported.
% We suggest that the plant-modeling community can adapt the methods in this thesis to verify the behavior of plant models.

\vspace{.5cm}
\textbf{Keywords:} Physical Reservoir Computing (PRC), Unconventional Computing, Plant Physiology, functional-structural plant models (FSPMs).

% - Physical reservoir computing is a machine learning method for performing computations with physical substrates.
% - PRC has seen succesful applications in e.g.\ compliant robots and photonic circuits.
% - Recently, Pieters (2022) proposed the use of plants as a reservoir medium.
% - In this work we set out to identify plant physiological processes that display emergent computational qualities.
% - More specifically we propose a framework to quantitatively compare the reservoir properties of plant physiological processes.

% - In this thesis we used mechanistic plant models as our test subjects
% - Specifically, we used a model of grapevine and a model of common wheat.
% - Four our experiments, we compare the properties of ..., ..., … and … .

% - Our experiments showed promising results for the future of plant reservoir computing.
% - However, the quality of our data was limited by the time resolution of the used plant simulations,
% - As a side effect, our framework was able to highlight flaws and unusual behavior in the used mechanistic models.