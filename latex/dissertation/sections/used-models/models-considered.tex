
% Introduction to the list of models.
% We searched for FSPMs that fit our criteria through Quantitative Plant and a literature search
There are diverse models in the literature, but many do not meet all our demands.
There are a few common shortcomings.
For one, many FSPMs only model a limited set of physiological processes \citep{louarn_two_2020}.
These simulations are too narrow to gain convincing evidence of reservoir dynamics present in real-world plants.
Other published works show sophisticated architectures and solid real-world validations but do not make their simulations open source \citep{prieto_functionalstructural_2020}. 


% Discussion of considered of FSPMs.
Table \ref{table:models-related-work} shows a selection of plant models we examined more closely.
We briefly investigated V-Mango \citep{boudon_v-mango_2020} and WALTer \citep{lecarpentier_walter_2019}. Ultimately, we disregarded these models because they model growth with time steps of one day. 
This step size is too large to explore reservoir dynamics at the time scales we want to observe (\SI{1}{s}-\SI{1}{h}).

PlaNet-Maize \citep{lobet_modeling_2014} and Soybean FSPM \citep{coussement_turgor-driven_2020} are growth models, but the step size of one hour warrants taking a deeper look;
when the growth rate is very slow compared to the time scale of the target task, the reservoir can be regarded as approximately static, as \citet{pieters_reservoir_2022} demonstrated with \textit{in vivo} strawberry plants.
Unfortunately, we could not further explore this direction due to practical reasons.
PlaNet-Maize is open-source, but the code repository contains no documentation on getting started or making modifications.
The soybean model is closed-source but was made available to us by co-author T. De Swaef.
Regrettably, we met many difficulties setting up the simulation environment, and the GroIMP platform made it unintuitive to implement adaptations as an untrained user.
Because of these reasons, combined with the limited time scope of this project, we did not explore these models further.


\begin{table}[!ht]
    \raggedright
    \caption{\acrshort{fspm}s we explored for use in RC experiments. Models highlighted in bold are used for the final results presented in this work.}
    \def\arraystretch{1}%  1 is the default, change whatever you need
    \begin{tabular}{
        >{\raggedright} m{0.166\linewidth}
        >{\raggedright\arraybackslash} m{0.14\linewidth} 
        >{\centering\arraybackslash} m{0.1\linewidth}
        >{\centering\arraybackslash} m{0.11\linewidth}
        >{\raggedright\arraybackslash} m{0.166\linewidth}
        >{\centering\arraybackslash} m{0.13\linewidth}
    }
        \toprule
        \textbf{Model} & \textbf{Crop} & \textbf{Growth} & \textbf{Time Step} & \textbf{Implementation} & \textbf{Open Source} \\ 
        \midrule
        PlaNet-Maize \citep{lobet_modeling_2014} & Maize & Yes & \SI{1}{h} & Java & Yes \\
        \arrayrulecolor{black!10!white}
        \midrule
        \textbf{CN-Wheat \citep{barillot_cn-wheat_2016}} & \textbf{Common wheat} & \textbf{No} & \textbf{1 h} & \textbf{Python (OpenAlea)} & \textbf{Yes} \\
        \midrule
        \textbf{HydroShoot \citep{albasha_hydroshoot_2019}} & \textbf{Common grapevine} & \textbf{No} & \textbf{1 h} & \textbf{Python (OpenAlea)} & \textbf{Yes} \\
        \midrule
        WALTer \citep{lecarpentier_walter_2019} & Common wheat & Yes & \SI{1}{d} & Python (OpenAlea) & Yes \\
        \midrule
        Soybean FSPM \citep{coussement_turgor-driven_2020} & Soybean & Yes & \SI{1}{h} & Java (GroIMP) & No \\
        \midrule
        V-Mango \citep{boudon_v-mango_2020} & Mango tree & Yes & \SI{1}{d} & Python (OpenAlea), R & Yes \\
        \midrule
        WheatFspm \citep{gauthier_functional_2020} & Common wheat & Yes & \SI{1}{h} & Python (OpenAlea) & Yes \\ 
        \midrule
        \citet{prieto_functionalstructural_2020} (unnamed) & Common grapevine & No & \SI{1}{h} & Python (OpenAlea) & No \\
        \arrayrulecolor{black}
        \bottomrule
    \end{tabular}
    \label{table:models-related-work}
\end{table}

% Discussion of shortlisting of FSPMs.
HydroShoot \citep{albasha_hydroshoot_2019} and a similar, unnamed model by \citet{prieto_functionalstructural_2020} are both FSPMs of grapevine; they have a static geometry and model a diverse set of physiological functions involved in gas exchange.
As far as we could find, \citeauthor{prieto_functionalstructural_2020} did not publish the code for their model.
In contrast, HydroShoot is open source and very well documented.
It was straightforward to set up using the Conda package manager for Python. The use of well-documented libraries from the OpenAlea platform made it easy to track the plant's state or adjust its environment.

Finally, WheatFspm \citep{gauthier_functional_2020} is another growth model.
WheatFspm builds upon the prior work CN-Wheat \citep{barillot_cn-wheat_2016}, a post-vegetative growth model of the same species.
Because we prefer (pseudo-)static models, we used CN-Wheat for our experiments instead.
The simulation code provided in the source repository already extracts all physiological time series, making CN-Wheat a very convenient model for our purposes.

Both HydroShoot and CN-Wheat have a good overlap in the physiological processes and meteorological inputs they simulate. 
These shared qualities enable a closer comparison of how the models behave as reservoirs when solving the same regression tasks.
The following section introduces the two selected models in more detail and compares them from a \acrshort{rc} perspective.
